\documentclass[ignorenonframetext]{beamer}
\usetheme{Madrid}
\usepackage[utf8x]{inputenc}
\usepackage{bortzmeyer-utils}

\title{The next step for DPRIVE}
\author{Stéphane Bortzmeyer - AFNIC}
\date{IETF 97 - Seoul}

\begin{document}

\begin{frame}
\maketitle  
\end{frame}

\begin{frame}
  \frametitle{Current state}
  \begin{itemize}
  \item<2->Charter said ``primary focus [\ldots] to develop mechanisms that
    provide confidentiality between DNS Clients and Iterative Resolvers''
   \item<3->RFC 7858 and
     \path{draft-ietf-dprive-dtls-and-tls-profiles} provides
     encryption and stub$\longrightarrow$resolver authentication
  \item<4->Charter also says ``may also later consider mechanisms that provide confidentiality
between Iterative Resolvers and Authoritative Server''
  \end{itemize}
\end{frame}

\begin{frame}
  \frametitle{Proposal}
  \begin{itemize}
  \item<2->Do not reinvent the wheel: reuse DNS-over-(D)TLS
  \item<3->This leaves the issue of \emph{authentication}
  \end{itemize}
\end{frame}

\begin{frame}
  \frametitle{Big difference}
  \begin{itemize}
    \item<2->stub$\longrightarrow$resolver: few servers, static config
      such as key pinning is OK
    \item<3->resolver$\longrightarrow$authoritative: many servers,
      need something more dynamic
    \end{itemize}
\end{frame}

\begin{frame}
  \frametitle{Possible techniques of authentication}
             [Should be all in \path{draft-bortzmeyer-dprive-step-2}]
   \begin{itemize}
   \item<2->Encode key in name (DNScrypt-style)
   \item<3->Regular PKIK validation based on DNSname
   \item<4->Key in DNS (DANE)
   \item<5->\ldots  
   \end{itemize}            
\end{frame}

\begin{frame}
  \frametitle{Tasks}
  \begin{itemize}
  \item<2->Decide on one (or several) solutions in
    \path{draft-bortzmeyer-dprive-step-2}
  \item<3->Write an I-D
  \item<4->Do we need to update the charter?  
  \end{itemize}
\end{frame}

\end{document}
