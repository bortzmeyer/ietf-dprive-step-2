\documentclass[ignorenonframetext]{beamer}
\usetheme{Madrid}
\usepackage[utf8x]{inputenc}
\usepackage{bortzmeyer-utils}

\title{Resolver to authoritative}
\author{Stéphane Bortzmeyer - AFNIC}
\date{IETF 101 - London}

\begin{document}

\begin{frame}
\maketitle  
\end{frame}

\begin{frame}
  \frametitle{Proposal}
  \begin{itemize}
  \item<2->DNS-over-(D)TLS
  \item<3->Authentication with DANE
  \item<4->\path{draft-bortzmeyer-dprive-resolver-to-auth}  
  \end{itemize}
\end{frame}

\begin{frame}
  \frametitle{Open questions}
  \begin{itemize}
  \item<2->Try TLS before looking up DANE: because port 853 may be
    blocked, and because we may use the TLS chain extension.
  \item<3->Happy eyeballs: 53 and 853 at the same time.  
  \item<4->Is it really sensible to mention strict mode? It is
    unrealistic today.
  \item<5->Mandating TLS 1.3?  
  \end{itemize}
\end{frame}

\begin{frame}
  \frametitle{Rechartering}
  \begin{itemize}
  \item<2->Charter said ``primary focus [\ldots] to develop mechanisms that
    provide confidentiality between DNS Clients and Iterative Resolvers''
  \item<3->Charter also says ``may also later consider mechanisms that provide confidentiality
between Iterative Resolvers and Authoritative Server''
  \end{itemize}
\end{frame}

\begin{frame}
  \frametitle{Tasks}
  \begin{itemize}
  \item<2->Too little activity until now. Is there still interest?
  \item<3->Update the charter?
  \end{itemize}
\end{frame}

\end{document}
